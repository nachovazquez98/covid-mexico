\begin{tabular}{lrr}
\toprule
{} &  SVM-l1 &  LR-l1 \\
\midrule
Age                         &   1.000 &  1.000 \\
Chronic Renal Insufficiency &   0.156 &  0.134 \\
Obesity                     &   0.085 &  0.085 \\
Immunosuppression           &   0.095 &  0.080 \\
Diabetes                    &   0.068 &  0.065 \\
Hypertension                &   0.051 &  0.045 \\
COPD                        &   0.055 &  0.035 \\
Tobacco Use                 &   0.006 &  0.006 \\
Cardiovascular Disease      &   0.005 & -0.000 \\
Other                       &  -0.004 & -0.006 \\
Asthma                      &  -0.018 & -0.013 \\
Pregnant                    &   0.004 & -0.054 \\
Gender                      &  -0.078 & -0.078 \\
RESULTADO\_3                 &  -0.380 & -0.450 \\
\bottomrule
\end{tabular}
